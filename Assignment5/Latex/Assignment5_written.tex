    \documentclass[a4paper,12pt,russian]{extreport}
     
    \usepackage{extsizes}
    \usepackage{cmap} % для кодировки шрифтов в pdf
    \usepackage[T2A]{fontenc}
    \usepackage[utf8]{inputenx}
   \usepackage{float}
    \usepackage[english,main=russian]{babel}
    
    
    \usepackage{pscyr}
    \usepackage{algpseudocode}%algorithms description
     
    \usepackage{graphicx} % для вставки картинок
    \usepackage{amssymb,amsfonts,amsmath,amsthm,amstext} % математические дополнения от АМС
    \usepackage{dsfont} % математические дополнения от АМС
    \usepackage{indentfirst} % отделять первую строку раздела абзацным отступом тоже
    \usepackage[usenames,dvipsnames]{color} % названия цветов
    \usepackage{makecell}
    \usepackage{multirow} % улучшенное форматирование таблиц
    %\usepackage{ulem} % подчеркивания(вместо обычного emph)
     
    \linespread{1} % полуторный интервал
    %\renewcommand{\rmdefault}{ftm} % Times New Roman
    \frenchspacing
    
    %Поля страницы
    \usepackage{geometry}
    \geometry{left=2cm}
    \geometry{right=2cm}
    \geometry{top=2cm}
    \geometry{bottom=2cm}
    
    %Название списка литературы
    \renewcommand\bibname{Литература}
    %\pagestyle{fancy}
    %\fancyhf{}
    %\fancyhead[R]{\thepage}
    %\fancyheadoffset{0mm}
    %\fancyfootoffset{0mm}
    %\setlength{\headheight}{17pt}
    %\renewcommand{\headrulewidth}{0pt}
    %\renewcommand{\footrulewidth}{0pt}
    %\fancypagestyle{plain}{ 
    %    \fancyhf{}
    %    \rhead{\thepage}}
    %\setcounter{page}{1} % начать нумерацию страниц с №5
    
    %О Рисунках и подписях
    \usepackage[tableposition=top]{caption}
    \usepackage{subcaption}
    \DeclareCaptionLabelFormat{gostfigure}{Fig. #2}
    \DeclareCaptionLabelFormat{gosttable}{Таблица #2}
    \DeclareCaptionLabelSeparator{gost}{~---~}
    \captionsetup{labelsep=gost}
    \captionsetup[figure]{labelformat=gostfigure}
    \captionsetup[table]{labelformat=gosttable}
    \renewcommand{\thesubfigure}{\asbuk{subfigure}}
    
    %Заголовки
    
    \usepackage{etoolbox}
	\makeatletter
	\patchcmd{\chapter}{\if@openright\cleardoublepage\else\clearpage\fi}{}{}{}
	\makeatother%to suppress newpages on chapters
	
    \usepackage{titlesec}
 
\titleformat{\chapter}
    {\Large\bfseries}
    {\thechapter}
    {8pt}{}   
\titlespacing*{\chapter}{0pt}{5pt}{25pt}%left-before-after, topskip для того, чтобы не было пустоты сверху

\titleformat{\section}
    {\large\bfseries}
    {\thesection}
    {1em}{}
\titlespacing*{\section}{0pt}{8pt}{20pt}
 
\titleformat{\subsection}
    {\normalsize\bfseries}
    {\thesubsection}
    {1em}{}


\begin{document}
\chapter*{ Assignment 5, Optimization methods (Maxim Kaledin)}

\section*{Problem 3}

Apply Sylvester's criterium to ensure that matrix is positive semidefinite. All principal minors should be non-negative, these are
\begin{align*}
-25y_1^2 \geq 0,\\
(4y_1+2)(-25y_1^2) \geq 0,\\
3y_2(4y_1+2) \geq 0,\\
3y_2 \geq 0,\\
4y_1+2 \geq 0.
\end{align*}
The first inequality gives $y_1=0$, and as result constraints are $D=\left\lbrace y: y_1=0,y_2 \geq 0\right\rbrace$ . Obviously, all feasible solutions are optimal and optimal goal is $0$.\\

Let $F_1,F_2,G$ are such symmetric matrices that the initial matrix $A=y_1F_1+y_2F_2+G$:
$$
F_1 =
\begin{bmatrix}
0 & 5 & 0\\
5 & 0 & 0\\
0 & 0 & 4
\end{bmatrix}, 
F_2 =
\begin{bmatrix}
0 & 0 & 0\\
0 & 3 & 0\\
0 & 0 & 0
\end{bmatrix}, 
G =
\begin{bmatrix}
0 & 0 & 0\\
0 & 0 & 0\\
0 & 0 & 2
\end{bmatrix}.
$$

Then we can write down the dual problem in the following way (denoting $Z\succeq 0$ as symmetric matrix of dual variables and scalar product as $(A,B)=\text{tr}(A^TB)$):

\begin{align*}
\max_{Z \succeq 0} - (Z,G), \quad s.t. \\
(Z,F_1)=2,
(Z,F_2)=0.
\end{align*}

Rewrite it in the coordinate form:

\begin{align*}
&\max_{Z \succeq 0} - 2z_{33}, \quad s.t. \\
&10z_{12}+4z_{33}=2,
&z_{22}=0.
\end{align*}

Now add Sylvester's criterium inequalities

\begin{align*}
&z_{11} \geq 0, ~ z_{22}\geq 0, ~ z_33 \geq 0,\\
&z_{11}z_{22}-z_{12}^2 \geq 0,\\
&z_{11}z_{33}-z_{13}^2 \geq 0,\\
&z_{22}z_{33} - z_{23}^2 \geq 0\\
&det(Z) \geq 0.
\end{align*}

This conditions can be significantly simplified if we add constraints from the dual problem. Finally, we obtain

\begin{align*}
&\max_{Z} - 2z_{33}, \quad s.t. \\
&10z_{12}+4z_{33}=2,\\
&z_{22}=0, z_{12}=0, z_{23}=0\\
&z_1 \geq 0, z_{33} \geq 0,\\
&z_{11}z_{33}-z_{13}^2 \geq 0.
\end{align*}

Since $z_{12}=0$ the entry $z_{33}=1/2$ and we can somehow assign other variables to satisfy feasibility. Thus the optimal dual goal is $-1$ which is not equal to optimal primal goal $0$.


\section* {Problem 4}

\subsection*{1}

The problem can be reformulated as
\begin{align*}
&\min_{x,t} t \quad s.t. \\
&t \geq a_{i}^Tx + b_i \quad \forall i \in \left\lbrace 1,..,m \right\rbrace.
\end{align*}

Corresponding Lagrange dual problem is
\begin{align*}
&\max_{\lambda} \inf_{x,t} \left[t+ \sum_{i=1}^{m} \lambda_i(a_{i}^Tx + b_i - t)\right],\\
& s.t. \quad \lambda \geq 0,
\end{align*}
where $\lambda_i \geq 0$ are dual variables. We can easily simplify this because it is a linear program:

\begin{align*}
&\quad \max_{\lambda} b^T\lambda,\quad s.t.\\
&\quad 
\begin{bmatrix}
A\\
-1...-1
\end{bmatrix}\lambda = 
\begin{bmatrix}
0\\
\vdots\\
0\\
-1\end{bmatrix}
\end{align*}

\subsection*{2}

Let 
$$
\hat{f}_2(y)=\frac{1}{\alpha}\log{\left( \sum_{i=1}^m \exp{\alpha y_i}\right)},
$$
and $y_i=a_i^Tx+b_i$ be new constraints. First, formulate the primal problem:

\begin{align*}
& \min_{x,y} \hat{f}_2(y)\\
&s.t. \quad y_i=a_i^Tx+b_i \quad \forall i \in \left\lbrace 1,..,m \right\rbrace.
\end{align*}

The dual one is defined as

\begin{align*}
&\max_{\lambda} \inf_{x,y} \left[\frac{1}{\alpha}\log{\left( \sum_{i=1}^m \exp{\alpha y_i}\right)} + \sum_{i=1}^{m} \lambda_i(a_{i}^Tx + b_i - y_i)\right]=\\
&=\inf_{y} \left[\frac{1}{\alpha}\log{\left( \sum_{i=1}^m \exp{\alpha y_i}\right)}-\sum_{i=1}^m\lambda_iy_i\right]+\inf_{x,y}\left[ \sum_{i=1}^{m} \lambda_i(a_{i}^Tx + b_i)\right].
\end{align*}


\subsection*{3}

Let us show the right side of inequality, in order to do that we can exploit monotonicity of exponent:

\begin{align*}
&\frac{1}{\alpha}\log\sum_{i=1}^m\exp\left(\alpha(a_{i}^Tx+b_i)\right)-\max_{i} (a_{i}^Tx+b_i) \leq \\
&\leq \frac{1}{\alpha}\log\left( m\exp\left(\alpha \max_{i}(a_i^Tx+b_i) \right )\right)-\max_{i} (a_{i}^Tx+b_i)=\\
&= \frac{\log(m)}{\alpha}.
\end{align*}



\section* {Problem 5}

Define Lagrange function of a problem $L(x,\lambda)=\Vert Ax-b\Vert_2^2+\sum_{i=1}^m\lambda_i(c_{i\cdot}x-d_i)$. Since there are no inequality constraints, KKT conditions are exactly
\begin{align*}
&\frac{\partial L}{\partial x} =0\\
&\frac{\partial L}{\partial \lambda} =0,
\end{align*}
where derivatives denote the gradients truncated to needed set of variables. These sets of constraints are exactly
\begin{align*}
&\frac{\partial L}{\partial x_j} = 2\sum_{i=1}^m a_{ij}\sum_{k=1}^n(a_{ik}x_k - b_i)+ \sum_{i=1}^p\lambda_i c_{ij}, \\
&Cx=d.
\end{align*}
Let us write the first expression shorter:
$$
\nabla_{x}L = 2A^T(Ax-b) +  C^T\lambda.
$$
Since the goal function is convex, solution to system
$$
\begin{cases}
&2A^T(Ax-b)+C^T\lambda=0,\\
&Cx=d
\end{cases}
$$
exists and is an optimal solution of primal problem. With pseudo-inverse (it exists since $C$ has full row rank):
\begin{align*}
& x=(C^TC)^{-1}C^Td,\\
& \lambda: \quad C^T\lambda = -2A^T(Ax-b).
\end{align*}

Now derive the dual problem:

$$
\max_{\lambda} \inf_{x} L(x,\lambda).
$$

Lagrangian $L$ remains convex for all $\lambda$, so the solution to inner problem is given by the system of normal equations

$$
2A^T(Ax-b)+C^T\lambda=0
$$
and can be found based on any $\lambda$ since matrix $A^TA$ has full rank. We could write the solution explicitly:
$$
x=(A^TA)^{-1}\left(-\frac{1}{2}C^T\lambda+A^Tb\right).
$$

\section*{Problem 6}

\subsection*{(1)}
Consider problem (4) firstly:

\begin{align*}
&\min_x c^Tx\\
&s.t.\quad Ax=b, ~x \geq 0.
\end{align*}

Its dual problem is 

\begin{align*}
&\max_{\lambda,\mu} \inf_x \left( c^Tx +\sum_{i=1}^m \lambda_i(a_{i\cdot}x-b_i) - \sum_{j=1}^n\mu_jx_j \right),\\
&s.t. \quad \mu\geq 0.
\end{align*}
It can be simplified to
\begin{align*}
&\max_{\lambda} b^T\lambda,\\
&s.t.~ A^Ty \leq c.
\end{align*}
KKT-conditions for primal problem are
\begin{itemize}
\item $c+A^T\lambda-\mu =0$ (stationarity, $\nabla_x L(x,\lambda,\mu)=0$),
\item feasibility conditions (as in problem statement),
\item $Ax=b$ (from $\nabla_\lambda L(x,\lambda,\mu)=0 $),
\item $\mu_jx_j=0 \quad \forall j$ (complementary slackness).
\end{itemize}

\subsection*{(2)}

The idea is the same. Consider problem (5):

\begin{align*}
&\min_x c^Tx - \tau\sum_{i=1}^n \log(x_i)\\
&s.t.\quad Ax=b.
\end{align*}

Its dual problem is (let $(1/x)$ denotes coordinate-wise division)

\begin{align*}
\max_{\lambda} \inf_x \left( c^Tx -\tau\sum_{i=1}^n \log(x_i) +\sum_{i=1}^m \lambda_i(a_{i\cdot}x-b_i) \right).
\end{align*}
and KKT-conditions are
\begin{itemize}
\item $c-\tau(1/x)+A^T\lambda =0$ (stationarity, $\nabla_x L(x,\lambda,\mu)=0$),
\item feasibility conditions (as in problem statement),
\item $Ax=b$ (from $\nabla_\lambda L(x,\lambda,\mu)=0 $).
\end{itemize}

Since the Hessian matrix of inf sub-problem is positive definite (it is even diagonal), infimum is achieved when neccessary conditions are satisfied (i.e. stationarity):

$$
(1/x)=\frac{1}{\tau}(c+A^T\lambda)=y(\lambda).
$$

So, simplified dual is

\begin{align*}
\max_{\lambda}  \left( c^T(1/y(\lambda)) +\tau\sum_{i=1}^n \log(y(\lambda)_i) +\sum_{i=1}^m \lambda_i(a_{i\cdot}(1/y(\lambda))-b_i) \right).
\end{align*}

After some manipulation we can obtain 

$$
x_i=\frac{\tau}{c_i+(a_{\cdot i})^T\lambda}
$$

and simplify the dual even more

\begin{align*}
\max_{\lambda}  \left( \tau\sum_{i=1}^n\frac{c_i}{c_i+(a_{\cdot i})^T\lambda} - n\tau\log\tau+\tau\sum_{i=1}^n \log(c_i+a_i^T\lambda) +\sum_{i=1}^m \lambda_i(a_{i\cdot}(1/y(\lambda))-b_i) \right).
\end{align*}

Consider closely the sum
\begin{align*}
&\sum_{i=1}^m \lambda_ia_{i\cdot}(1/y(\lambda))=\sum_{i=1}^m \lambda_i \sum_{j=1}^n \frac{a_{ij}\tau}{c_j+(a_{\cdot j})^T\lambda}=\\
&= \sum_{j=1}^n \sum_{i=1}^m \frac{\lambda_ia_{ij}\tau}{c_j+(a_{\cdot j})^T\lambda}=\sum_{j=1}^n \frac{(a_{\cdot j})^T\lambda \tau}{c_j+(a_{\cdot j})^T\lambda}.
\end{align*}

It cancels with the first sum and results in $m\tau$. Finally, the dual problem is
\begin{align*}
\max_{\lambda}  \left( \tau\sum_{j=1}^n \log(c_j+(a_{\cdot j})^T\lambda) - b^T\lambda \right).
\end{align*}

\end{document}

